\chapter{Conclusion} % Conclusion
\label{ch:sum}
The \textbf{Log Analyzer} in its current state is capable of investigating a DSP application implemented
using \textbf{PipeRT} framework. The project offers a general framework for analyzing, the analyzer
can be extended with different measurements and checkers based on the user's or the developer's needs.
The already implemented measurements and checkers show different sides of the application's pipeline
and check its sanity.

The \textbf{Log Analyzer} helps to experiment with the \textbf{PipeRT} on different DSP applications
by providing a user-friendly graphical interface to ease the assessment of such experiments.

The future of the project is very promising mainly because of the architectural decisions
that give a lot of room for new features and logic to be added, to
give a better user experience, or to add a new concept to the project. A few future additions can be:
\begin{itemize}
    \item Adding more visualizations for the graphs, for example, pie charts.
    \item Extending checkers to check the pipeline and to associate that with the pipeline's measurement.
    \item Storing the old packet cycles information to show them to the user if requested.
    \item Comparing different packet cycles visually and statistically.
\end{itemize}