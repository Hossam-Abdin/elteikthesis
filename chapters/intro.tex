\chapter{Introduction} % Introduction
\label{ch:intro}

\section{Motivation}
Logging is not an easy addition to any system but becomes useful only
with a tool that knows how to extract valuable data from a huge
stream and from these data can bring an overview, and
statistics that describe the behavior and analyze it. The log analyzer
became essential not only to support such a system and shows its flows,
but also to visualize a picture to force the developer(s) to see what
he/she never expected.

For all the reasons mentioned and more, building a log 
analyzer to show the bottlenecks and help the developers of DSP (Digital signal processing)
applications who are using the PipeRT framework was a project eager to be born.

\section{Thesis Structure}
This thesis is composed of 4 main chapters, a bibliography, 
a list of figures, and a list of codes.

Chapter 2 is going to introduce the user documentation, including
how to install and run the analyzer.

Chapter 3 contains the developer documentation with detailed Structure
of the implementation, and its capabilities to be extended.

Chapter 4 is the conclusion, and the summary of the project can be
found there, with the future work of the project.